\setcounter{chapter}{7}
\chapter{Confidence Intervals}

\section[Conditions]{Appropriate conditions to construct a confidence interval}

Read Section 6.1: Infrence for a single proportion, Section 6.1.1: Identifying when the sample proportion is nearly normally and Section 6.1.2: Confidence intervals for a proportion, in the Open Intro Statistics book, pages 275-276.  

\subsection[CI for Mean Conditions]{Conditions associated with a confidence interval for population means}

For additional reading on when it is appropriate to use confidence interval for means, read Section 5.3.1: Conditions for using the \(t)\)-distribution for inference on a sample mean, in the Open Intro Statistics book, page 223

\subsection[CI for Proportion Conditions]{Conditions associated with a confidence interval for population proportion}

For additional reading on when it is appropriate to use confidence interval for means, read in the Open Intro Statistics book.

\section[CI as Hypothesis Test]{Perform a hypothesis test using a confidence interval}

....in the Open Intro Statistics book.

\section[CI for Difference of Means]{Construct and interpret a confidence interval about the difference between two population means using two independent samples.}

Read Section 5.3: Difference of two means and Section 5.3.1: Confidence interval for a difference of means, in the Open Intro Statistics book, pages 230-233.  

After reading Sections 5.3 and 5.3.1 in the Open Intro Statistics book, go to the end of the chapter and try homework questions:

\section[CI for Difference of Proportions]{Construct and interpret a confidence interval about the difference between two population proportions using two independent samples.}

Read Section 6.2: Difference of two proportions, Section 6.2.1: Sample distribution of the differnece of two proportions and Section 6.2.2: Confidence intervals for \(p_{1}-p_{2}\), in the Open Intro Statistics book, pages 280-281.  

After reading Sections 6.2, 6.2.1 and 6.2.2 in the Open Intro Statistics book, go to the end of the chapter and try homework questions:

\section{Construct and interpret a confidence interval about the difference between two population means using paired samples.}

Read Section 5.2: Paired data and Section 5.2.1: Paired observations, pages 228-229 in the Open Intro Statistics book.

After reading Sections 5.2 and 5.2.1, in the Open Intro Statistics book, go to the end of the chapter and try homework questions:

\section{Technology with confidence intervals}

     \subsection{Input two independent samples and execute the commands to construct a two-sample difference of means confidence interval and interpret the output}

     \subsection{Input two independent samples and execute the commands to construct a two-sample difference of proportions confidence interval and interpret the output}

     \subsection{Input two paired samples and execute the commands to construct a one-sample confidence interval and interpret the output}
