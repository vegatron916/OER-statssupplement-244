\setcounter{chapter}{2}
\chapter{Probability}

\section{Bayes Theorem}

The tree diagram in Figure \ref{fig:treeMechanics} shows a situation where an individual leaves their car at a mechanic's shop. There are three mechanics: Miranda, Jose, and Tanya. Miranda is the most experienced mechanic, while Tanya is the least experienced. Miranda repairs 55\% of the cars; Jose repairs 30\% of the cars; and Tanya repairs the remaining 15\% of the cars. Miranda's repairs fail for 5\% of cars on which she works; Jose's fail for 10\%; and Tanya's repairs fail for 20\%.

\begin{figure}
\centering
\includegraphics[width=4in]{images/treeMechanics.png}
\caption{Tree diagram for repairs at the mechanic's shop.}
\label{fig:treeMechanics}
\end{figure}

This presents an interesting application of conditional probability. Using the tree diagram above we can compute some probabilities.

\begin{example}
Selecting one car at random that has been repaired at this shop, use the tree diagram to compute the probability that
\begin{enumerate}
 \item Miranda repaired the car, \(P(M)\),
 \item the repair failed if Tanya worked on the car, \(P(F|T)\),
 \item Jose repaired the car and the repair was successful, \(P(J\text{ AND }S)\).
\end{enumerate}

Notice that we can work top to bottom through the ``branches'' of the tree diagram to consider different probability questions. The top branch contains the ``simple'' probabilities of Miranda, Jose, and Tanya repairing a randomly selected car, like \(P(M)\) above. Working down, the second branch contains conditional probabilities of failure or success, depending on which mechanic performed the repair like \(P(F|T)\) above. We multiply simple and conditional probabilities along the tree diagram's branches to calculate joint probabilities of various outcomes, like \(P(J\text{ AND }S)\) above. With a complete tree diagram, we can ask and answer questions like those above quickly.
\end{example}

But what about probability questions that the diagram does not answer directly? 

\begin{example}
Selecting one car at random, what is the probability that Tanya repaired the car, if that repair failed, \(P(T|F)\)? This question ``reverses the condition'' from conditional probability problem above. Though \(P(T|F)\) is not in the diagram, we can use the conditional probability formula as follows:

\[ P(T|F) = \frac{P(T \text{ AND } F)}{P(F)} \]

Now from the tree diagram, we see
\[P(T \text{ AND } F) = P(T) \cdot P(F|T) = (0.15)(0.20) = 0.03.\]
Since there are three mechanics, we can calculate \(P(F)\):
\begin{align*}
P(F) &= P(M \text{ AND } F) + P(J \text{ AND } F) + P(T \text{ AND } F) \\
        &= (0.55)(0.05) + (.30)(0.10) + (0.15)(0.20) \\
        &= 0.0875
\end{align*}
Then we have \(P(T|F) = \frac{P(T \text{ AND } F)}{P(F)} = \frac{0.03}{0.0875} \approx 0.343\), so there is a 34.3\% chance that if a random repair failed, Tanya did that repair.
\end{example}

Bayes Theorem is a mathematical formula that encodes this process. In this case, \emph{Bayes Theorem} combines our work above into a formula like so:

\[
P(T|F) = \frac{P(T) \cdot P(F|T) }{  P(M \text{ AND } F) + P(J \text{ AND } F) + P(T \text{ AND } F) }
\]
Computation using this formula is as complicated as the computations we did above using the tree diagram to guide our work. 

At the introductory level, it is wise to focus on using the tree diagram and the conditional probability formula to ``reverse the condition,'' rather than to worry about memorizing and using the Bayes Theorem formula.

\section{Calculate and interpret marginal distribution}

Consider the contingency table shown below for the number of males and females in a class using an OER (Open Educational Resource) or a textbook.

\begin{center}
\begin{tabular}{l|cc}
& OER & Textbook \\ \hline
Male & 10 & 25 \\
Female & 20 & 5	
\end{tabular}
\end{center}

Since this table shows counts, we would also refer to it as a frequency table.  Since the totals for each sex and for each class type are also of interest, we can include a row and a column for the respective totals, as shown below.

\begin{center}
\begin{tabular}{l|cc|c}
& OER & Textbook & Total \\ \hline
Male & 10 & 25 & 35 \\
Female & 20 & 5 & 25 \\ \hline
Total & 30 & 30 & 60
\end{tabular}
\end{center}

The totals end up in the margins of the contingency table. The Total column shows the \emph{marginal distribution} of sex, and the Total row shows the \emph{marginal distribution} of class type. 

The contingency table can also be represented using proportions as shown below.

\begin{center}
\begin{tabular}{l|cc|c}
& OER & Textbook & Total \\ \hline
Male & 0.17 & 0.42 & 0.58 \\
Female & 0.33 & 0.08 & 0.42 \\ \hline
Total & 0.50 & 0.50 & 1
\end{tabular}
\end{center}

Here we can see the relative marginal distributions of sex and of class type in the Total column and row, respectively.

\section{Homework}
\begin{enumerate}
 \item In some states, police used to set up DUI checkpoints where they would stop drivers and try to determine if each driver had been drinking.  Assume on a given night 15\% of people are drinking. If you have been drinking, police will stop you and give you a breath check 85\% of the time. If you have not been drinking, police will stop you and give you a breath check 10\% of the time.
 \begin{enumerate}
  \item Create a tree diagram for this scenario. Use \(B = \) Breath check and \(D = \) Drinking.
  \item What is the probability that you are drinking, given that you did have a break check, that is, find \(P(D|B^{c})\).
 \end{enumerate}
 \item Confidence is of the utmost importance for soccer players.  A good player will score on 80\% of their penalty kicks.  If they score on their first penalty kick, there is a 90\% chance they will score on a second penalty kick.  If they miss their first penalty kick, there is a 70\% chance they will not score on a second penalty kick.

\begin{center}
 \begin{enumerate}
  \item Create a tree diagram for this scenario. Use \(S =\) scores and \(S^{c} =\) Does not score.
  \item Find \(P(S)\)
  \item Find \(P(S \text{ 2nd} | S^{c} \text{ 1st})\)
  \item Find \(P(S \text{ 1st} | S^{c} \text { 2nd})\)
 \end{enumerate}
\end{center}

 \item A food packaging factory has both machines and workers on assembly line.  First, machines do part of the packaging and then the worker finishes it by hand.  There is a 1\% chance a machine makes a defect in the package.  If the machine does make a defect, there is a 70\% chance the worker also makes a defect.  If the machine does not make a defect, there is a 20\% chance the worker makes a defect.
 \begin{enumerate}
  \item Create a tree diagram for this scenario. Use \(M =\) Machine Defect and \(W =\) Worker Defect.
  \item Find \(P(M \text{ AND } W)\)
  \item Find \(P(W|M)\)
  \item What is the probability the machine made a defect given that a worker made a defect?
 \end{enumerate}

 \item A computer company is investigating its sales. Customers buy laptops or desktops, which are either new or used. The company knows about 10\% of people buy new laptops, about 60\% of people buy laptops and about 30\% of people buy new computers.
 \begin{enumerate}
 \item Complete the two way table below.
  
  \begin{center}
  \begin{tabular}{l|cc|c}
  & New & Used & Total \\ \hline
  Laptop & 0.10 & & 0.60 \\
  Desktop & & & \\ \hline
  Total & 0.30 & & 1
  \end{tabular}
  \end{center}
  
 \item Give the marginal distribution of computer style.
 \item Find \( P(\text{Laptop AND Used}) \).
 \item Find \( P( \text{ New OR Desktop } ) \).
 \item Find \( P( \text{Laptop} | \text{New} ) \).
 \end{enumerate}
 \item Consider the table below for products at a grocery store.

\begin{center}
 \begin{tabular}{l|cc|c}
  & Sale item & Non-sale item & Total \\ \hline
  Organic &  & &  \\
  Non-organic & & & \\ \hline
  Total &  & & 
 \end{tabular}
\end{center}
  \begin{enumerate}
  \item Assume 40\% of products are organic, 20\% of products are non-organic and on sale, and 90\% of products are neither on sale nor organic.  Complete the two way table.  
  \item Find \( P(\text{Non-organic}) \).
  \item Find \( P(\text{Non-organic AND Non-sale}) \).
  \item Find \( P(\text{Organic} | \text{Sale}) \).
  \end{enumerate}
\end{enumerate}