\documentclass[12pt]{book}
\title{MTH 244 Supplement to OpenStax College \emph{Introductory Statistics}}
%\author{Emiliano Vega \& Ralf Youtz}
%\date{\today}

\usepackage{amsmath,
			amsfonts,
			amssymb,
			amsthm,
			mleftright}
\usepackage[per-mode=symbol]{siunitx}
\usepackage[super]{nth}
\newtheorem{example}{Example}

\usepackage{enumitem}
% set the second-level and third-level enumerated items
% to format automatically
%\setlist[enumerate,1]{label*=(\Alph*)}
%\setlist[enumerate,2]{label*=\Roman*.}
%\setlist[enumerate,3]{label*=\arabic*.}
% or you could create your own list environment, separate from the enumerate environment
\newlist{mylist}{enumerate}{5}
\setlist[mylist,1]{label*=\arabic*.}
\setlist[mylist,2]{label*=\arabic*.}
\setlist[mylist,3]{label*=\arabic*.}
\DeclareMathOperator{\SE}{SE}   %THIS IS AMAZING---the thing I never looked up that I've been waiting for!!!!

\usepackage{xcolor}    %color options
\usepackage{blindtext} %use to check formatting

\usepackage{fancyhdr}		%fancy header/footer ready to go
\lhead{\em\small\rightmark}			% left header
%\chead{\small Youtz}					% central header
\rhead{\em\small\leftmark}		% right header
%\lfoot{}								% left footer
\lfoot{You can find the \emph{Open Intro Statistics} text for free at \url{https://www.openintro.org/download.php?file=os3_tablet&referrer=/stat/textbook.php}.}			
\cfoot{}			% central footer
%\rfoot{}								% right footer

%\usepackage[left=1.5 in,	
%			right=1.5 in, 
%			top=1.5 in, 
%			bottom=1.5 in]{geometry} %sets margins
% paragraph decisions:
%\setlength{\parskip}{0.5 pc}     %space between paragraphs; 	pc is pica; pt is point
%\setlength{\parindent}{2 em}     %indent space;  		em is em-space; ex is ex-space

\usepackage{multicol}		% creates environmnets with columns
\setlength{\columnsep}{2pc}     % distance between columns
\setlength{\columnseprule}{0pt} % width of line separating columns

\usepackage{calc}		% allows measure calculation for minipage dimension, etc.

%%%%%%%%%%% TABLE STUFF STARTS HERE %%%%%%%%%%%%%%%

\usepackage{multirow}

\usepackage{array}					% needed for next command
%\setlength{\extrarowheight}{0.5pc}		% adds 0.5pc height to tabular rows

%\setlength{\tabcolsep}{1cm} 	%buffer space on left and right of tabular columns

\usepackage{colortbl,xcolor}		% allows table coloring

\usepackage{caption}				% more captioning options
%\captionsetup{font=sc} 					% small capital letters
%\captionsetup{labelfont=bf,textfont=it} 	% bold label, italic caption
\captionsetup{labelfont=sc}

\usepackage{subfig} 				% allows ``subfloats''

\usepackage{booktabs}			% better table design options

\usepackage[colorlinks]{hyperref}

%%%%%%%%%%%%%%% MODULE 5: FIGURES PACKAGES HERE %%%%%%%%%%%%%%%%%%%%%
\usepackage{graphicx}	% graphics options
\usepackage{tikz}
\usetikzlibrary{arrows}
\usepackage{pgfplots}
%\pgfplotsset{compat=1.11}     %Use the command \pgfplotsset{} in the preamble to globally set all of the axis options in the examples in subsection pgfplots template.

%\newcommand{\addition}[3]{#1+#2=#3}

\begin{document}
\pagestyle{fancy}

\frontmatter

\maketitle

%\chapter{License}

\copyright\ 2016 \quad Portland Community College
\begin{center}
\includegraphics[width=1in]{images/open-educational-resources.jpg}
\end{center}
This work is licensed under a Creative Commons Attribution-ShareAlike 4.0 International License. 

\noindent\textbf{You are free to:}
\begin{description}
 \item[Share] --- copy and redistribute the material in any medium or format
 \item[Adapt] --- remix, transform, and build upon the material
for any purpose, even commercially.
\end{description}
\noindent\textbf{Under the following terms:}
\begin{description}
 \item[Attribution] --- You must give \emph{appropriate credit}, provide a link to the license, and \emph{indicate if changes were made}. You may do so in any reasonable manner, but not in any way that suggests the licensor endorses you or your use.
 \item[ShareAlike] --- If you remix, transform, or build upon the material, you must distribute your contributions under the \emph{same license} as the original.
\end{description}

\textbf{No additional restrictions} --- You may not apply legal terms or \emph{technological measures} that legally restrict others from doing anything the license permits.

\noindent Complete license information at\\
\url{https://creativecommons.org/licenses/by-sa/4.0/legalcode}



\chapter{Preface}

This MTH 244 Supplement was produced with grant support from 
Open Oregon 
for PCC students using 
\emph{Introductory Statistics} 
by Barbara Illowsky and Carol Dean
from
\url{https://openstax.org/}.

For complete PCC Course Content Outcome Guide, see \url{https://www.pcc.edu/ccog/default.cfm?fa=ccog&subject=MTH&course=244}.

\begin{center}
\includegraphics[width=3in]{images/Open-Oregon-for-lights.png}
\end{center}


%\tableofcontents
%\listoffigures
%\listoftables

\mainmatter

\setcounter{chapter}{7}
\chapter{Confidence Intervals}

\section[Conditions]{Appropriate conditions to construct a confidence interval}

Read Section 6.1: Infrence for a single proportion, Section 6.1.1: Identifying when the sample proportion is nearly normally and Section 6.1.2: Confidence intervals for a proportion, in the Open Intro Statistics book, pages 275-276.  

\subsection[CI for Mean Conditions]{Conditions associated with a confidence interval for population means}

For additional reading on when it is appropriate to use confidence interval for means, read Section 5.3.1: Conditions for using the \(t)\)-distribution for inference on a sample mean, in the Open Intro Statistics book, page 223

\subsection[CI for Proportion Conditions]{Conditions associated with a confidence interval for population proportion}

For additional reading on when it is appropriate to use confidence interval for means, read in the Open Intro Statistics book.

\section[CI as Hypothesis Test]{Perform a hypothesis test using a confidence interval}

....in the Open Intro Statistics book.

\section[CI for Difference of Means]{Construct and interpret a confidence interval about the difference between two population means using two independent samples.}

Read Section 5.3: Difference of two means and Section 5.3.1: Confidence interval for a difference of means, in the Open Intro Statistics book, pages 230-233.  

After reading Sections 5.3 and 5.3.1 in the Open Intro Statistics book, go to the end of the chapter and try homework questions:

\section[CI for Difference of Proportions]{Construct and interpret a confidence interval about the difference between two population proportions using two independent samples.}

Read Section 6.2: Difference of two proportions, Section 6.2.1: Sample distribution of the differnece of two proportions and Section 6.2.2: Confidence intervals for \(p_{1}-p_{2}\), in the Open Intro Statistics book, pages 280-281.  

After reading Sections 6.2, 6.2.1 and 6.2.2 in the Open Intro Statistics book, go to the end of the chapter and try homework questions:

\section{Construct and interpret a confidence interval about the difference between two population means using paired samples.}

Read Section 5.2: Paired data and Section 5.2.1: Paired observations, pages 228-229 in the Open Intro Statistics book.

After reading Sections 5.2 and 5.2.1, in the Open Intro Statistics book, go to the end of the chapter and try homework questions:

\section{Technology with confidence intervals}

     \subsection{Input two independent samples and execute the commands to construct a two-sample difference of means confidence interval and interpret the output}

     \subsection{Input two independent samples and execute the commands to construct a two-sample difference of proportions confidence interval and interpret the output}

     \subsection{Input two paired samples and execute the commands to construct a one-sample confidence interval and interpret the output}


%\include{hypothesisTestingWithTwoSamples}

%\include{theChiSquareDistribution}

\setcounter{chapter}{13}
\chapter{ANOVA}

\section{Use a multiple comparisons method to determine which pairs of means differ; interpret the results}

Read section 5.5.5: Multiple comparisons and controlling Type \(1\) Error rate, in the Open Intro Statistics book, pages 253-256.

Although honestly we don't do this...

\section{Identify Marginal Probabilities} 

Do we need to do this?  Feels like we already do it in MTH 243



\setcounter{chapter}{12}
\chapter{Simple Linear Regression And Correlation}

\section{Check the conditions associated with constructing  a least-squares linear regression model, and construct such a model}

Read 7.2.2: Conditions for the least squares line, in the Open Intro Statistics book, page 341

\section{Specify the probability distribution of the random error term, and estimate the standard deviation of this distribution.}

I actually don't know what this means?

\section{Construct and interpret a confidence interval to estimate the slope of the population regression model.}

NOT in Open Intro

\section{Technology for the least-squares line for estimation and prediction}

     \subsection{Construct and interpret a residual plot} 

Read Section 7.1.3: Residuals, in the Open Intro Statistics book
€‹ \subsection{Construct and interpret a confidence interval for the mean value of the response variable when the explanatory variable takes on a specific value.}
     \subsection{Using technology, construct and interpret a prediction interval for an individual value of the response variable when the explanatory variable takes on a specific value.}
     \subsection{Using technology, input sample data and execute the commands to produce a least-squares regression equation, a fitted line, a residual plot, and \(r^2\); interpret the output.}

\end{document}